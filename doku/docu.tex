\documentclass[a4paper,12pt]{report}
\usepackage[utf8]{inputenc}
\usepackage[T1]{fontenc}
\usepackage[ngerman]{babel}
\usepackage{graphicx}
\usepackage{listings}
\usepackage{amsmath}
\usepackage{hyperref}

\hypersetup{
    colorlinks=true,
    linkcolor=black,
    urlcolor=blue,
    citecolor=black,
    pdfborder={0 0 0}
}

\usepackage{caption}
\usepackage{geometry}
\usepackage{lmodern}
\geometry{left=3cm,right=2.5cm,top=2.5cm,bottom=2.5cm}

% Titelblatt
\begin{document}

\begin{titlepage}
    \centering
    \vspace*{3cm}
    {\Huge \textbf{Dokumentation der praktischen Arbeit}}\\[2cm]
    {\large zur Prüfung zum \\ Mathematisch-technischen Softwareentwickler}\\[2cm]
    {\Large \textbf{Thema: <Thema einfügen>}}\\[1cm]
    {16.05.2025}\\[5cm]
    {Daniel Ebel}\\[2cm]
    Prüflingsnummer: <Nummer einfügen>\\[0.5cm]
    Bearbeitungszeitraum: 12.05.2025 - 16.05.2025\\[0.5cm]
    Programmiersprache: Java\\[0.5cm]
    Ausbildungsort: INFORM GmbH
\end{titlepage}

% Eigenständigkeitserklärung
\chapter*{Eigenständigkeitserklärung}
Ich erkläre, dass das vorliegende Prüfprodukt von mir selbstständig erstellt wurde. Die als Arbeitshilfe genutzten Unterlagen sind in der Arbeit vollständig aufgeführt. Ich versichere, dass der vorgelegte Ausdruck mit dem Inhalt der von mir erstellten digitalen Version identisch ist. Weder ganz noch in Teilen wurde die Arbeit bereits als Prüfungsleistung vorgelegt. Mir ist bewusst, dass jedes Zuwiderhandeln als Täuschungsversuch zu gelten hat, der die Anerkennung als Prüfungsleistung ausschließt.

\tableofcontents
\listoffigures

\chapter{Aufgabenanalyse}
\section{Beschreibung Eingabedatei}
<Inhalt einfügen>

\section{Verarbeitung}
<Inhalt einfügen>

\section{Beschreibung Ausgabedatei}
<Inhalt einfügen>

\section{Simulationsende}
<Inhalt einfügen>

\chapter{Verfahrensbeschreibung}
\section{Nebenläufigkeit}
<Inhalt einfügen>

\section{Einlesen}
<Inhalt einfügen>

\section{Mathematische Methoden}
\subsection{Umrechnung und Normierung}
<Inhalt einfügen>

\subsection{Glättung der Daten}
<Inhalt einfügen>

\subsection{Obere Einhüllende}
<Inhalt einfügen>

\subsection{Pulsbreite}
<Inhalt einfügen>

\section{Ausgeben}
<Inhalt einfügen>

\chapter{Programmbeschreibung}
\section{Datenhaltung}
<Inhalt einfügen>

\section{Einlesen}
<Inhalt einfügen>

\section{Verarbeiten}
<Inhalt einfügen>

\section{Ausgeben}
<Inhalt einfügen>

\section{Gesamtablauf}
<Inhalt einfügen>

\chapter{Simulationsergebnis}
\section{Signalverarbeitung}
<Inhalt einfügen>

\section{Nebenläufigkeit}
<Inhalt einfügen>

\chapter{Zusammenfassung und Ausblick}
\section{Zusammenfassung}
<Inhalt einfügen>

\section{Ausblick}
<Inhalt einfügen>

\appendix
\chapter{Abweichungen und Ergänzungen zum Vorentwurf}
<Inhalt einfügen>

\chapter{Benutzeranleitung}
<Inhalt einfügen>

\chapter{Entwicklungsumgebung}
<Inhalt einfügen>

\chapter{Quellcode}
%\lstinputlisting[language=Java]{<Pfad/zu/Datei.java>}

\end{document}
