\chapter{Erweiterbarkeit}

Eine vorstellbare weitere Anforderung wäre, 
dass Fahrzeuge nicht überholen können, sondern einen Mindestabstand zum vorausfahrenden Fahrzeug einhalten müssen.
Diese zusätzliche Anforderung ließe sich durch Anpassen der DirectedEdgeInfo Klasse und der updateVehicle Methode erreichen.
Die Fahrzeuge würden nicht mehr zentral in der vehicles Liste gespeichert werden, sondern in der DirectedEdgeInfo Klasse in einer
Queue Struktur. Neue Fahrzeuge auf der Straße würden ans Ende der Queue eingefügt werden und Fahrzeuge, die die Straße verlassen, 
würden, würden am Anfang der Queue entfernt werden.
Beim Aktualisieren der Fahrzeuge in der simulate() Methode würde über die DirectedEdgeInfos in der directedEdges HashMap iteriert werden
und die Position des Fahrzeugs, welches am nächsten am Verlassen der Straße ist, würde als erstes aktualisiert werden, also das Fahrzeug, welches am Anfang der 
Straßenqueue ist. Bei dem ersten aktualisierten Fahrzeug würde berechnet werden, wo ein Fahrzeug, welches den Minimalabstand zu dem Fahrzeug am Anfang der Queue einhält, wäre und diese Position wird an
das zweite Fahrzeug weitergegeben. Nun wird die Position des zweiten Fahrzeuges so angepasst, dass dieses entweder normal weiterfährt, wenn es nicht den Minimalabstand unterschreiten würde oder,
wenn es den Minimalabstand unterschreiten würde die Position auf die übergebene zuvor berechnete Position gesetzt.
Am Ende aller Iterationen aller DirectedEdges, werden die Positionen der Fahrzeuge neu berechnet, welche Straßen an Kreuzungen gewechselt haben.

Wenn der Verkehrsfluss an Kreuzungen über Ampeln geregelt werden würde, würde die Position von Fahrzeugen,
die eine Straße an einer Kreuzung wechseln würde um die Anzahl an Takten, wie lange die Ampel gehen würde gestoppt, bevor diese weiterfahren dürfen.

Beim Hinzufügen von Kreisverkehren wäre eine ähnliche Logik anwendbar. Ein Fahrzeug würde für die Taktlänge eines im Stadtplan durch ein extra Zeichen angegebenen Kreisverkehr
an der Kreuzung verweilen, bevor diese weiterfährt.

Für das Implementieren einer Rechts vor Links Überprüfung müsste man für jedes Fahrzeug, welches eine Kreuzung passieren würde erstmal prüfen,
welche Straßen rechts sind. Dafür würde man die xy Komponenten des Anfangs End Vektors der gerade gefahrenen Straße tauschen und danach die y Komponente negieren,
um den links orthogonalen Vektor zu dem gerade gefahrenen Straßenvektor zu erhalten. Nun findet man alle directedEdges bei denen to der Kreuzung entspricht an der abgebogen werden soll.
Von diesen sind die Straßen rechts, bei denen das Skalarprodukt des links orthogonalen Vektor mit dem Anfangs End Vektor der potentiellen rechts Straße größer 0 ist.

Jetzt muss berechnet werden, ob mehrere Autos in einem vorher festgelegten Zeitraum an wie oben beschriebenen Straßen an einer Kreuzung stehen.
Für Stadtpläne mit niedrigen Einfallspunkttaktraten könnte es jedoch zu deadlocks führen, wenn an einer Kreuzung jedes Auto auf das Auto rechts warten muss.

Mehrere Fahrbahnen pro Strecke erfordern eine Lockerung des Einleseformats,
so müssen doppelte Einfallspunktzeilen, wo sich lediglich die Taktrate unterscheidet zugelassen werden. 
