\chapter{Einleitung}
Zur Bewertung von Verkehrsflüssen und zur Identifikation kritischer Straßenabschnitte soll eine Verkehrssimulation entwickelt werden.
Grundlage der Simulation ist eine Eingabedatei mit Einfalls- und Kreuzungspunkten sowie ein definierter Simulationszeitraum.
Fahrzeuge treten an Einfallspunkten mit bestimmten Taktraten auf, bewegen sich mit zufälliger,
individuell festgelegter Geschwindigkeit durch das Netz und treffen an Kreuzungen zufallsbasiert Abbiegeentscheidungen gemäß festgelegter Wahrscheinlichkeiten.

Die Eingabe erfolgt über eine Textdatei mit folgendem Beispielinhalt:

\begin{lstlisting}
# Beispielhausen
Zeitraum:
50 1

Einfallspunkte:
A 0 0 B 2
C 0 2 B 5
D 4 0 E 3
F 4 2 E 2
G 5 1 E 3

Kreuzungen:
B 0 1 A 20 C 30 E 50
E 4 1 D 20 F 20 G 10 B 50
\end{lstlisting}

Die erste Zeile gibt einen Kommentar wieder.
Danach folgen der Simulationszeitraum (hier 50 Sekunden mit Ausgaben alle 1 Sekunde),
gefolgt von Einfallspunkten. Jeder Einfallspunkt enthält Position, Zielkreuzung und Taktzeit.
Anschließend werden Kreuzungen definiert,
die jeweils eingehende Straßen mit Richtungsverteilungen enthalten.

Ziel der Simulation ist es, Fahrzeugbewegungen zu modellieren,
deren Positionen für jeden Zeitschritt in einer Datei ausgegeben und mit einem Visualisierungstool dargestellt werden.
Zusätzlich werden für jede Straße statistische Kenngrößen wie kumulierte Fahrzeuganzahl und maximale Streckenauslastung berechnet.