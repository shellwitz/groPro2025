\chapter{Abweichungen vom ersten Entwurf}
Bei der Implementierung wurden einige Abweichungen vom ersten Entwurf vom Montag, den 12.05.2025, vorgenommen.
So wurde eine \texttt{CityDTO}-Datentransferklasse geschrieben, die verwendet wird, um ein \texttt{City}-Objekt zu instanziieren.
Dies ist angenehmer als die Nutzung eines Konstruktors mit fünf Eingabeparametern und erfordert keine Implementierung von
fünf \texttt{get}-Methoden in der \texttt{TextFileReader}-Klasse.

Die Klasse \texttt{TrafficNode}, die eine Kreuzung darstellen sollte, erhielt den passenderen Namen \texttt{Intersection}.
Es wurde eine Helper-Klasse mit einer statischen \texttt{getId()}-Methode implementiert, um die Fahrzeug-IDs zu generieren.

Im UML-Klassendiagramm der \texttt{Vehicle}-Klasse wurde das Attribut \texttt{double velocity} (für Geschwindigkeit) vergessen (existiert in der finalen Implementierung).
Anfangs wurde nicht spezifiziert, dass die Ausgabe in den Dateien \texttt{Plan.txt} und \texttt{Statistik.txt} nach alphabetisch sortierten Eingabeortsnamen erfolgen soll.
Deswegen wurde die Methode \texttt{getAlphabeticallySortedDirectedEdges()} geschrieben.

Die Methode \texttt{updateVehicle()} erhält einen \texttt{Iterator<Vehicle>} anstelle eines einzelnen \texttt{Vehicle}-Objekts, um ein sicheres Entfernen zu gewährleisten,
z. B. wenn ein Fahrzeug den Stadtplan über einen Einfallspunkt verlässt, während die Fahrzeugliste iteriert wird.

Die Methode \texttt{considerVehiclesThatAreStillDriving()}, die ursprünglich dazu gedacht war, Fahrzeuge,
die nach Ende der Simulationszeit noch unterwegs sind, in die Statistik aufzunehmen,
erwies sich als unnötig, da die Statistikberechnung Fahrzeuge bereits als den Abschnitt gefahren zählt,
sobald sie auf den jeweiligen Abschnitt auffahren.

Eine Reihe von \texttt{get}-Methoden wurde hinzugefügt, um den Outputwritern \\
(z. B. \texttt{PlanWriter}) die benötigten Informationen wie die \texttt{directedEdges} bereitzustellen.

Einige Anforderungen vom Montag wurden revidiert. So wurde ursprünglich festgelegt, dass jede Kreuzung mindestens drei Abbiegepunkte haben soll.
Da dies die Eingabe des „IHK-Beispiels Tunneldorf mit Umfahrung über I“ ungültig machen würde, wurde davon abgewichen,
sodass jede Kreuzung nun nur mindestens zwei verbindende Straßen haben muss.
In der Anfangsspezifikation wurde fälschlicherweise angenommen, dass die relativen Wahrscheinlichkeiten einer Kreuzung in
Prozent angegeben sind und sich zu 100 aufaddieren. Dies wurde geändert, sodass alle relativen Wahrscheinlichkeiten intern normiert werden.



