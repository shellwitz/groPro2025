\chapter{Tests}

Das Verifizieren, ob die Ausgabedateien korrekt sind,
lässt sich aufgrund der zufälligen Natur der Simulation und der fehlenden Vergleichsmöglichkeiten mit erwartetem Output
nur visuell über das PLot.py Skript.

Die Normalfälle von der IHK Website getestet und visuell als passend empfunden.
Ein Fokus wurde auf das Testen von Fehler und Grenzfällen bei der Einlese gelegt.

Hierfür wurde ein Ordner namens badInputTests erstellt und

\section{Ausführen der Tests}

\subsection{BadInputTestRunner}

Bei der Arbeit an dem TextFileReader wurde mehrfach refactored und die Bedingungsprüfungen wurden nacheinander implementiert und nach jeder Bedingung geprüft, 
ob frühere Fehlerfälle immer noch die erwartete Ausgabe liefern. 
Um diesen Prozess zu automatisieren, wurde das \texttt{BadInputTestRunner}-Testprogramm für Regressionstests geschrieben.

Der Quellcode ist die gleichnamige Java-Datei im \texttt{tests}-Ordner, ebenfalls im \\
\texttt{trafficsimulation}-Package. 
Das Programm führt die folgenden Schritte aus:

\begin{itemize}
    \item \textbf{Verzeichnisprüfung:} Es wird geprüft, ob das Verzeichnis \texttt{badInputTests} existiert und Eingabedateien enthält.
    \item \textbf{Definition der Testfälle:} Für jede fehlerhafte Eingabedatei wird die erwartete Fehlermeldung definiert. Beispiele für Fehler sind:
    \begin{itemize}
        \item Fehlende Einfallspunkte (\texttt{ERROR\_NO\_ENTRY\_POINTS}),
        \item Ungültige Koordinaten (\texttt{ERROR\_INVALID\_COORDINATE\_COMPONENT})
    \end{itemize}
    \item \textbf{Testausführung:} Jede Eingabedatei wird mit dem \texttt{TextFileReader} eingelesen. Falls die Datei fehlerhaft ist, wird eine Ausnahme ausgelöst.
    \item \textbf{Ergebnisprüfung:} Die tatsächliche Fehlermeldung wird mit der erwarteten Fehlermeldung verglichen. Stimmen sie überein, gilt der Test als bestanden.
    \item \textbf{Endergebnis:} Am Ende wird eine Übersicht ausgegeben, ob alle Tests bestanden wurden oder ob Fehler aufgetreten sind.
\end{itemize}


Zum Testen des Programms steht je ein Skript für Windows und Linux/macOS zur Verfügung. Unter Windows wird das Skript \texttt{test\_run.bat} verwendet.
Es führt die Datei \texttt{MyProgram.jar} für jede Datei im Ordner \texttt{input\_files} einzeln aus und übergibt sie als Argument an das Programm.

Zur Ausführung müssen alle Eingabedateien im Verzeichnis \texttt{input\_files} liegen und sich die \texttt{.jar}-Datei im gleichen Ordner wie das Skript befinden.
Das Skript kann entweder per Doppelklick oder über die Eingabeaufforderung mit \texttt{test\_run.bat} gestartet werden.

Für Linux oder macOS wird das Bash-Skript \texttt{test\_run.sh} genutzt. Auch hier wird die \texttt{MyProgram.jar} für jede Datei im Ordner \texttt{input\_files} aufgerufen.

Vor dem ersten Ausführen muss dem Skript Ausführungsrecht gegeben werden, z.\,B.\ durch den Befehl \texttt{chmod +x test\_run.sh}. Danach kann es mit \texttt{./test\_run.sh} ausgeführt werden.
Alternativ kann das Skript auch ohne Ausführungsrecht mit \texttt{bash test\_run.sh} gestartet werden.